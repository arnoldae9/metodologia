%%%%%%%%%%%%%%%%%%%%%%%%%%%%% Define Article %%%%%%%%%%%%%%%%%%%%%%%%%%%%%%%%%%
\documentclass{article}
%%%%%%%%%%%%%%%%%%%%%%%%%%%%%%%%%%%%%%%%%%%%%%%%%%%%%%%%%%%%%%%%%%%%%%%%%%%%%%%

%%%%%%%%%%%%%%%%%%%%%%%%%%%%% Using Packages %%%%%%%%%%%%%%%%%%%%%%%%%%%%%%%%%%
\usepackage[margin = 1 in ]{geometry}
\usepackage{graphicx}
\usepackage{amssymb}
\usepackage{amsmath}
\usepackage{amsthm}
\usepackage{empheq}
\usepackage{mdframed}
\usepackage{booktabs}
\usepackage{lipsum}
\usepackage{graphicx}
\usepackage{color}
\usepackage{psfrag}
\usepackage{pgfplots}
\usepackage{bm}
\usepackage[spanish]{babel}
\usepackage{natbib}
\usepackage{hyperref}
\usepackage{hyperref}
\hypersetup{
    colorlinks=true,
    linkcolor=blue,
    urlcolor=red,
    pdftitle={How to write a thesis},
    }
\usepackage{fancyvrb}
\usepackage{fancyhdr, lastpage}
\pagestyle{fancy}
\lhead{Universidad Autónoma de Nuevo León}
\rhead{Investigación de revistas}
%%%%%%%%%%%%%%%%%%%%%%%%%%%%%%%%%%%%%%%%%%%%%%%%%%%%%%%%%%%%%%%%%%%%%%%%%%%%%%%

% Other Settings

%%%%%%%%%%%%%%%%%%%%%%%%%% Page Setting %%%%%%%%%%%%%%%%%%%%%%%%%%%%%%%%%%%%%%%
\geometry{a4paper}

%%%%%%%%%%%%%%%%%%%%%%%%%% Define some useful colors %%%%%%%%%%%%%%%%%%%%%%%%%%
\definecolor{ocre}{RGB}{243,102,25}
\definecolor{mygray}{RGB}{243,243,244}
\definecolor{deepGreen}{RGB}{26,111,0}
\definecolor{shallowGreen}{RGB}{235,255,255}
\definecolor{deepBlue}{RGB}{61,124,222}
\definecolor{shallowBlue}{RGB}{235,249,255}
%%%%%%%%%%%%%%%%%%%%%%%%%%%%%%%%%%%%%%%%%%%%%%%%%%%%%%%%%%%%%%%%%%%%%%%%%%%%%%%

%%%%%%%%%%%%%%%%%%%%%%%%%% Define an orangebox command %%%%%%%%%%%%%%%%%%%%%%%%
\newcommand\orangebox[1]{\fcolorbox{ocre}{mygray}{\hspace{1em}#1\hspace{1em}}}
%%%%%%%%%%%%%%%%%%%%%%%%%%%%%%%%%%%%%%%%%%%%%%%%%%%%%%%%%%%%%%%%%%%%%%%%%%%%%%%

%%%%%%%%%%%%%%%%%%%%%%%%%%%% English Environments %%%%%%%%%%%%%%%%%%%%%%%%%%%%%
\newtheoremstyle{mytheoremstyle}{3pt}{3pt}{\normalfont}{0cm}{\rmfamily\bfseries}{}{1em}{{\color{black}\thmname{#1}~\thmnumber{#2}}\thmnote{\,--\,#3}}
\newtheoremstyle{myproblemstyle}{3pt}{3pt}{\normalfont}{0cm}{\rmfamily\bfseries}{}{1em}{{\color{black}\thmname{#1}~\thmnumber{#2}}\thmnote{\,--\,#3}}
\theoremstyle{mytheoremstyle}
\newmdtheoremenv[linewidth=1pt,backgroundcolor=shallowGreen,linecolor=deepGreen,leftmargin=0pt,innerleftmargin=20pt,innerrightmargin=20pt,]{theorem}{Theorem}[section]
\theoremstyle{mytheoremstyle}
\newmdtheoremenv[linewidth=1pt,backgroundcolor=shallowBlue,linecolor=deepBlue,leftmargin=0pt,innerleftmargin=20pt,innerrightmargin=20pt,]{definition}{Definition}[section]
\theoremstyle{myproblemstyle}
\newmdtheoremenv[linecolor=black,leftmargin=0pt,innerleftmargin=10pt,innerrightmargin=10pt,]{problem}{Problem}[section]
%%%%%%%%%%%%%%%%%%%%%%%%%%%%%%%%%%%%%%%%%%%%%%%%%%%%%%%%%%%%%%%%%%%%%%%%%%%%%%%

%%%%%%%%%%%%%%%%%%%%%%%%%%%%%%% Plotting Settings %%%%%%%%%%%%%%%%%%%%%%%%%%%%%
\usepgfplotslibrary{colorbrewer}
\pgfplotsset{width=8cm,compat=1.9}
%%%%%%%%%%%%%%%%%%%%%%%%%%%%%%%%%%%%%%%%%%%%%%%%%%%%%%%%%%%%%%%%%%%%%%%%%%%%%%%

%%%%%%%%%%%%%%%%%%%%%%%%%%%%%%% Title & Author %%%%%%%%%%%%%%%%%%%%%%%%%%%%%%%%
\title{Tarea 2}
\author{Investigación de la metodología en revistas}
\bibliographystyle{unsrtnat}

%%%%%%%%%%%%%%%%%%%%%%%%%%%%%%%%%%%%%%%%%%%%%%%%%%%%%%%%%%%%%%%%%%%%%%%%%%%%%%%

\begin{document}
    \maketitle
    \thispagestyle{empty}

    \newpage
    \section*{MHsalud}
    La primera de nuestras referencias es: \cite{mhsalud} la cual pueden visitar en la página: 
    \href{https://www.revistas.una.ac.cr/index.php/mhsalud}{MHsalud}.
    \newline
    Esta revista está orientada a la medicina, una de las características principales que me 
    llamaron la atención fue el hecho que entre los formatos que piden para entregar no viene 
    admitido pdf ni archivos tipo tex. 
    \\ Respecto a la estructura del manuscrito la letra se pide tipo Times New Roman, tamaño 12 
    interlineado sencillo entre otros detalles.
    \\ En la primera página se deben de integrar título del artículo (centrado, en mayúscula y 
    negrita). Debajo aparecerán los nombres y apellidos de los autores, grado académico, la 
    institución en la que elaboro la obra (lugar de afiliación) y el correo electrónico del 
    autor principal. 
    \\ Después se detallan de manera general la forma en que se deben de escribir las notas de
    pie (solo de ser necesarias) y se da una idea fundamental de la estructura del manuscrito.
    \\ En caso de ser un artículo se da una detallada explicación de cada sección, siendo 
    el primero {\bfseries Introducción, Metodología, Resultados, Discusión y Conclusiones}, cada uno de estos
    es detallado, pero nos interesa ver el apartado de Metodología.
    \\ En este último  mencionado, se detalla el uso de cuatro secciones:
    \begin{enumerate}
        \item Participantes.
        \item Instrumentos.
        \item Procedimiento.
        \item Análisis estadístico.
    \end{enumerate}
    El apartado que más me dio interés fue el de participantes, ya que repite el hecho que si se usan personas o 
    animales se debe de establecer las normas éticas seguidas o el mencionar el comité ético que aprobó dicho 
    experimento. \footnote{Como nota, es importa recordar que es una revista médica}
    \section*{Nature}
    Para nuestra siguiente referencia utilizaremos \cite{nature} la cual se puede visitar en el siguiente link: 
    \href{https://www.nature.com/nature/for-authors/formatting-guide}{Nature}.
    \\ En esta revista un detalle en el apartado de Métodos es la petición de no debe incluir figuras ni tablas, 
    estas mismas deben ser adjuntadas en un apartado de datos ampliados o información complementaria. \\
    A contrario de la anterior referencia, no se pide un tipo de letra o estructura específica, sin embargo si se 
    es insistente con que debe de tener un detalle corto y preciso de los procesos y protocolos que se tomaron 
    para poder facilitar la reproducción de dicho experimento, y además se pide que los métodos vengan separados 
    por secciones en negritas y subsecciones que deberán ir enumeradas cada una de manera ascendente y desde la 
    última utilizada; por otra parte se recomienda usar un medio (no específica cual) para compartir datos, protocolos 
    ó resultados, inclusive en esta misma página se facilita uno al usuario.
    \section*{PNAS}
    En la siguiente referencia \cite{PNAS} cuyas siglas significa: Procedings of the National Academy of Sciences of
    the Unites States of America, a simple vista la página no proporciona detalles sobre la letra o estructura del 
    manuscrito o de las secciones pero tiene un apartado que acepta archivos tipo tex, y no solo eso tiene un apartado 
    donde puedes descargar el código  para un manuscrito, tanto del archivo principal como las clases, la bibliografía, 
    las ecuaciones y además te dice que hacer al obtener posibles errores de compilación en tu archivo y un extra es 
    que si usar Overleaf tiene ciertas reglas de como compartirlo desde esta página. \\
    Algo más que tiene es que tiene un pequeño cuadro con los tipos de documentos y que apartados debe de tener. 
    \section*{SCIELO}
    La penúltima de nuestras referencias es la Ingeniería Investigación y Tecnología de la  \cite{scielo}, la principal causa de 
    agregar esta referencia fue que es una revista de origen mexicano, en esta página 
    \href{http://www.scielo.org.mx/revistas/iit/einstruc.htm}{scielo}  se puede verificar la estructura del manuscrito
    a grandes rasgos no se define una letra o interlineado específico, sin embargo son muy precisos en el orden de las 
    secciones, otro detalle importante es que no aceptan archivos tipo tex y se pide que el documento no exceda las 
    6000 palabras o 15 páginas, sobre la metodología no se especifíca que debe de cumplir solo se pide que todos los
    métodos y pruebas tienen que estar presentes.
    \section*{Indian Journal of Pharmacology}
    Por último, quise agregar la referencia  \cite{international2006uniform} ya que me gustó mucho el indice que tiene la página, prácticamente te 
    detalla cada una de las partes que te pide, sin embargo al igual que la primer referencia esta más vinculada a la 
    medicina, en esta página \href{https://ijp-online.com/article.asp?issn=0253-7613;year=2006;volume=38;issue=2;spage=149;epage=162;aulast=International#IV.%20Manuscript%20preparation%20and%20submission}{IJP}
    tenemos al lado izquierdo un detalle de cada una de las secciones que debe de llevar así como unos apartados interesantes
    como la ética, publicaciones, referencias, figuras, tablas entre muchas más.
    \\ En el apartado de preparación de Manuscrito podemos encontrar algunos lineamientos generales, en la sección de 
    Métodos hay un detalle que me llamó la atención, resulta que si uno o varios de los autores utilizan como variable
    la raza o la etnia de sus participantes tienen que detallar y justificar el uso de estas mismas.
    Además se pide de igual manera que en otras referencias el uso detallado para describir los métodos y procedimientos
    de tal manera que permita de manera eficaz y rápida replicar estos mismo, y añadiendo los detalles que se contemplaron 
    para localizar, seleccionar, extraer y sintetizar datos, y por último se pide que cada uno de ellos lleve datos estadísticos
    que se detallan en el apartado de estadísticos.









    \newpage
    \bibliography{biblio.bib}
\end{document}