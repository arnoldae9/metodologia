\documentclass{report}
\usepackage[margin = 1 in]{geometry}
\usepackage[spanish]{babel}
\usepackage[utf8]{inputenc}
\usepackage{amsmath, amsfonts, amsthm, graphicx, lipsum}
\usepackage{hyperref}
\hypersetup{
    colorlinks=true,
    linkcolor=blue,
    urlcolor=red,
    pdftitle={Insert tittle of pdf},
    }
\usepackage{fancyvrb}
\usepackage{fancyhdr, lastpage}
\pagestyle{fancy}
\lhead{Tarea 9 Investigación de repositorios y mailing list}
\rhead{Universidad Autónoma de Nuevo León}
% \cfoot{Page \thepage\ of \pageref{LastPage}}

\usepackage{etoolbox} %Use carefully!
\patchcmd{\chapter}{\thispagestyle{plain}}{\thispagestyle{fancy}}{}{}

\usepackage[Glenn]{fncychap}
%Options: Sonny, Lenny, Glenn, Conny, Rejne, Bjarne, Bjornstrup
\usepackage[margin = 1 in]{geometry}

\usepackage{xcolor}
\usepackage{tikz}
\usepackage[most]{tcolorbox}

\newtcbtheorem{theo}%
  {Theorem}{}{theorem}
  
\usepackage{siunitx}
\usepackage{setspace}
\onehalfspacing
\begin{document}

\section*{Pre-Prints}

\subsection*{OSFPREPRINTS}
En este sitio podemos encontrar una biblioteca gratuita de artículos no publicados o artículos en revisión, así mismo podemos aportar nuestros propios artículos.

Un detalle importante es que cuenta con una guía para el llenado de la petición y un ejemplo de como realizar un artículo, este mismo se puede descargar en formato pdf.

Podemos ingresar a esta página con en este sitio \href{https://osf.io/preprints/}{OSF}.

\subsection*{UANL}

Nuestra universidad cuenta con un sistema denominado auto-archivar, el cual consta de poder guardar nuestros artículos de forma que podamos distribuir copias digitales con el fin de dar acceso al trabajo.

Se puede visitar su página, así como su guía en el siguiente link \href{http://eprints.uanl.mx}{UANL}

\subsection*{Repositorio Nacional}

Una alternativa es el repositorio nacional el cual es una plataforma digital que proporciona acceso abierto en texto completo a diversos recursos de información académica, científica y tecnológica, es decir, sin requerimientos de suscripción, registro o pago.

Se puede visitar la página en el siguiente enlace \href{https://www.repositorionacionalcti.mx}{Repositorio Nacional} y aquí mismo descargar la solicitud para depositar nuestros documentos.

\subsection*{Scielo}
En este repositorio podemos pedir adjuntar nuestros resultados, algo importante a resaltar es que se nos puede pedir archivos que avalen nuestros resultados.

En esta misma página se nos presentan dos guías para la petición y carga de documentos, todos los que se puedan pedir. Se puede consultar en \href{https://scielo.org/es/}{Scielo}.

\subsection*{Latindex}
Tal vez una de las más exigentes que encontré, no solo se pide que accedas a licencias de acceso abierto, además de esto se pide llenar un registro en el cual se pide que tengamos al menos una revista publicada y además la revista en cuestión tiene que estar con registro ISSN.

Tanto como los ejemplos y procesos a seguir se pueden consultar en el siguiente link \href{https://www.latindex.org/latindex/inicio}{Latindex}

\section*{Mailing List}
\subsection*{SIAM}
El primer mailing-list es \href{http://lists.siam.org/mailman/listinfo/siam-opt}{SIAM} el cual es un grupo enfocado en la optimización, en su página describe que no esta enfocado en un campo específico es de manera general y se estarán recibiendo correos tanto de debates, reuniones, ponencias etc.

\subsection*{Yale}
Este mailing-list lo adjunto por la importancia de la universidad \href{https://subscribe.yale.edu/login?returnURL=https%3A%2F%2Fsubscribe.yale.edu%2Fbrowse%3Fsearch%3Dmath%2Bdus}{Yale}, es un grupo que esta enfocado en las matemáticas, tiene acceso tanto para alumnos y público en general.

\subsection*{DMANET}
Discrete mathematics and algorithms es un grupo que esta enfocado en como su nombre lo deja claro en matemáticas y algoritmos, es de acceso a todo público y se puede consultar en su página \href{http://www.zaik.uni-koeln.de/mailman/listinfo/dmanet}{DMANET}.

\subsection*{Source Forges}
En este grupo se describe el enfoque en la optimización de redes, también es de acceso público y se puede consultar en su página \href{https://sourceforge.net/projects/opennop/lists/opennop-devel}{Source Forge}.

\subsection*{Stanford}
Una lista muy interesante es la de la \href{https://mailman.stanford.edu/mailman/listinfo/icme-linear-algebra-optimization}{Stanford}, es un grupo enfocado en \textit{Linear-algebra-optimization} es de acceso público.


\end{document}