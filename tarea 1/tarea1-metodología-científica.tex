\documentclass[a4paper, 12pt]{article}
%\usepackage[spanish]{babel}
\usepackage[utf8]{inputenc}
\usepackage{hyperref}
\hypersetup{
	colorlinks=true,
	linkcolor=black,
	urlcolor=blue,
	pdftitle= Tittle of research proporsal
}



\usepackage[ letterpaper, margin=1in ]{ geometry }

\usepackage{fancyvrb}
\usepackage{fancyhdr, lastpage}
\pagestyle{fancy}
\lhead{Complete Name of University}
\rhead{Complete Name of Faculty}
\cfoot{Page \thepage\ of \pageref{LastPage}}
\lfoot{Footnotes pages}
\rfoot{\today}

\usepackage{setspace}
\onehalfspacing

\newcommand{\Tittle}[1]{ {\Huge \bfseries{ #1 }}  }
\newcommand{\subTittle}[1]{ {\Large \bfseries{ #1 }}  }
\newcommand{\normalTittle}[1]{ {\normalsize \bfseries{ #1 }}  }

\usepackage{biblatex}

\bibliography{bibliotarea1}
\addbibresource{bibliotarea1.bib}


\begin{document}
\ttfamily %mecanografiada
%\rmfamily %Roman
%\sffamily %sans
\begin{center}

	\Tittle{Title of research proporsal} \par
	\subTittle{Name of autor}  \par
	\subTittle{Supervised for:} \par
	\subTittle{Name of you supervisor }\textbf{\Large lastname} \newline
\end{center}

\begin{abstract}
	It's a a brief summary of approximately 300 words. It includes the important questions, the rationale for the study, the hypothesis, the method and the other characteristics. When describing the method, the design, procedure, results, and discussion must be included.
\end{abstract}


\section{Objetives}
Describe clearly and concisely the objective of your research proporsal.
\section{Please give a brief justification of your proposed research project:}
In this section the research proposal must be justified.
\section{Introduction}
The formal research programmes in the educational institutes are meant to train students to practice
research as a profession.The main purpose of the introduction is to provide the necessary background or
context for your research problem. How to frame the research problem is perhaps
the biggest problem in proposal writing.
\section{Literature  Review}
Sometimes the literature review is incorporated into the introduction section.
However, most professors prefer a separate section, which allows a more thorough
review of the literature.
\newline Import points:
\begin{enumerate}
	\item Organization and structure.
	\item Focus, unity and coherence.
	\item Not be repetitive or verbose.
	\item Falling cite.
	\item Citing irrelevant or trivial references.
	\item Not depending too much on secundary sources.
\end{enumerate}

\section{Methods}
The Method section is very important because it tells your Research Committee how
you plan to tackle your research problem. It will provide your work plan and describe
the activities necessary for the completion of your project.
You need to demonstrate your knowledge of alternative methods and make the case
that your approach is the most appropriate and most valid way to address your
research question. \newline
For quantitative studies, the method section typically consists of the following
sections:
\begin{enumerate}
	\item Design.
	\item Subjects or participants.
	\item Instruments.
	\item Procedure.

\end{enumerate}

\section{Results}
Obviously you do not have results at the proposal stage. However, you need to have
some idea about what kind of data you will be collecting, and what statistical
procedures will be used in order to answer your research question or test you
hypothesis.
\section{Discussion}
t is important to convince your reader of the potential impact of your proposed
research. You need to communicate a sense of enthusiasm and confidence without
exaggerating the merits of your proposal. That is why you also need to mention the
limitations and weaknesses of the proposed research, which may be justified by time
and financial constraints as well as by the early developmental stage of your
research area.
\section*{Common Mistakes in Proposal Writing}
\begin{enumerate}
	\item Failure to provide the proper context to frame the research question.
	\item Failure to delimit the boundary conditions for your research.
	\item Failure to cite landmark studies.
	\item Failure to accurately present the theoretical and empirical contributions by other researchers.
	\item Failure to stay focused on the research question.
	\item Failure to develop a coherent and persuasive argument for the proposed research.
	\item Too much detail on minor issues, but not enough detail on major issues.
	\item Too much rambling -- going "all over the map" without a clear sense of direction. (The best proposals move forward with ease and grace like a seamless river.)
	\item Too many citation lapses and incorrect references.
	\item Too long or too short.
	\item Failing to follow the APA style.
	\item Slopping writing.
\end{enumerate}
\section*{Contact}

E-mail: \href{youremail@mail.com}{email} \newline
Fax: 555-555-555-555\newline
Phone 1:132-465-4659\newline
Phone 2:132-486-49661 \newline
Facebook: \href{https://www.facebook.com/}{my name in facebook}
%puedes agrear el campo que desear utilizando href o ref



\newpage
\nocite{york}
\nocite{wong2016write}
\nocite{sudheesh2016write}
\nocite{nte2006research}
\nocite{iqbal2007learning}
\printbibliography

\end{document}
%bibliografia 2: http://www2.psych.utoronto.ca/users/shkim/How%20to%20Write%20a%20Research%20Proposal.pdf
% bibliografia 3: https://www.ncbi.nlm.nih.gov/pmc/articles/PMC5037942/
%bibliografia https://www.ajol.info/index.php/njm/article/view/37249/25850
% biliografia 5:  https://www.researchgate.net/profile/Javed-Saani/publication/228983837_Learning_from_a_Doctoral_Research_Project_Structure_and_Content_of_a_Research_Proposal/links/53f55f8c0cf2888a7491bf23/Learning-from-a-Doctoral-Research-Project-Structure-and-Content-of-a-Research-Proposal.pdf
